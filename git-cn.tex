%!TEX TS-program = xelatex
%!TEX encoding = UTF-8 Unicode
\documentclass[landscape,a4paper]{cheatsheet}
\usepackage[english]{babel}
\usepackage[utf8]{inputenc}
\usepackage{float}
\usepackage{graphics}
\usepackage{xeCJK}
\setmainfont{Times New Roman} %缺省英文字体
\setCJKmainfont[BoldFont=SimHei,ItalicFont=KaiTi]{宋体}
\setCJKsansfont{黑体}%serif是有衬线字体sans serif无衬线字体。
\setmonofont{宋体} % 等寬字型
\XeTeXlinebreaklocale "zh"
\XeTeXlinebreakskip = 0pt plus 1pt minus 0.1pt

\setCJKfamilyfont{song}{SimSun} %宋体 song
\newcommand{\song}{\CJKfamily{song}} % 宋体 (Windows自带simsun.ttf)
\setCJKfamilyfont{xs}{NSimSun} %新宋体 xs
\newcommand{\xs}{\CJKfamily{xs}}
\setCJKfamilyfont{fs}{FangSong} %仿宋 fs
\newcommand{\fs}{\CJKfamily{fs}} %仿宋体 (Windows自带simfs.ttf)
\setCJKfamilyfont{kai}{KaiTi} %楷体2312 kai
\newcommand{\kai}{\CJKfamily{kai}}
\setCJKfamilyfont{yh}{Microsoft YaHei} %微软雅黑 yh
\newcommand{\yh}{\CJKfamily{yh}}
\setCJKfamilyfont{hei}{SimHei} %黑体 hei
\newcommand{\hei}{\CJKfamily{hei}} % 黑体 (Windows自带simhei.ttf)
\setCJKfamilyfont{msunicode}{Arial Unicode MS} %Arial Unicode MS: msunicode
\newcommand{\msunicode}{\CJKfamily{msunicode}}
\setCJKfamilyfont{li}{LiSu} %隶书 li
\newcommand{\li}{\CJKfamily{li}}
\setCJKfamilyfont{yy}{YouYuan} %幼圆 yy
\newcommand{\yy}{\CJKfamily{yy}}
\setCJKfamilyfont{xm}{MingLiU} %细明体 xm
\newcommand{\xm}{\CJKfamily{xm}}
\setCJKfamilyfont{xxm}{PMingLiU} %新细明体 xxm
\newcommand{\xxm}{\CJKfamily{xxm}}

\setCJKfamilyfont{hwsong}{STSong} %华文宋体 hwsong
\newcommand{\hwsong}{\CJKfamily{hwsong}}
\setCJKfamilyfont{hwzs}{STZhongsong} %华文中宋 hwzs
\newcommand{\hwzs}{\CJKfamily{hwzs}}
\setCJKfamilyfont{hwfs}{STFangsong} %华文仿宋 hwfs
\newcommand{\hwfs}{\CJKfamily{hwfs}}
\setCJKfamilyfont{hwxh}{STXihei} %华文细黑 hwxh
\newcommand{\hwxh}{\CJKfamily{hwxh}}
\setCJKfamilyfont{hwl}{STLiti} %华文隶书 hwl
\newcommand{\hwl}{\CJKfamily{hwl}}
\setCJKfamilyfont{hwxw}{STXinwei} %华文新魏 hwxw
\newcommand{\hwxw}{\CJKfamily{hwxw}}
\setCJKfamilyfont{hwk}{STKaiti} %华文楷体 hwk
\newcommand{\hwk}{\CJKfamily{hwk}}
\setCJKfamilyfont{hwxk}{STXingkai} %华文行楷 hwxk
\newcommand{\hwxk}{\CJKfamily{hwxk}}
\setCJKfamilyfont{hwcy}{STCaiyun} %华文彩云 hwcy
\newcommand{\hwcy}{\CJKfamily{hwcy}}
\setCJKfamilyfont{hwhp}{STHupo} %华文琥珀 hwhp
\newcommand{\hwhp}{\CJKfamily{hwhp}}

\setCJKfamilyfont{fzsong}{Simsun (Founder Extended)} %方正宋体超大字符集 fzsong
\newcommand{\fzsong}{\CJKfamily{fzsong}}
\setCJKfamilyfont{fzyao}{FZYaoTi} %方正姚体 fzy
\newcommand{\fzyao}{\CJKfamily{fzyao}}
\setCJKfamilyfont{fzshu}{FZShuTi} %方正舒体 fzshu
\newcommand{\fzshu}{\CJKfamily{fzshu}}

\setCJKfamilyfont{asong}{Adobe Song Std} %Adobe 宋体 asong
\newcommand{\asong}{\CJKfamily{asong}}
\setCJKfamilyfont{ahei}{Adobe Heiti Std} %Adobe 黑体 ahei
\newcommand{\ahei}{\CJKfamily{ahei}}
\setCJKfamilyfont{akai}{Adobe Kaiti Std} %Adobe 楷体 akai
\newcommand{\akai}{\CJKfamily{akai}}

\title{\texttt{Git} for gits}
\author{A. C. Hinrichs}
\date{\today}

\newcommand{\highlight}[1]{{\textsf{\color{primaryColor}#1}}}
\begin{document}
\maketitle

\begin{figure}[H]
  \centering
  \resizebox{0.75\linewidth}{!}{\includegraphics{pictures/git_2x.png}}
  \caption{Anleitung für diese Anleitung\protect\footnotemark}
  \label{fig:anleitung}
\end{figure}
\footnotetext{Quelle: \url{https://xkcd.com/1597/}}
\section{Basis-Workflow}
中文 \textit{中文} \textbf{中文} \hwl{中文} Ein generischer Git-workflow, ein \highlight{Repository} muss
logischerweise nur ein mal angelegt werden, 
\subsection{Ein \highlight{Repository} klonen}
\begin{lstlisting}[language=bash]
  $ git clone <REPO-URL>
\end{lstlisting} % $ %Hack for Math-highlighting
Git erstelt nun eine \highlight{Arbeitskopie} des entferneten
Repos. Diese liegt anschließend in einem Unterorder des aktiven
Verzeichnisses. 

Exisitert kein entferntes Repository kann man das aktive
verzeichniss mittels
\begin{lstlisting}[language=bash]
  $ git init
\end{lstlisting} % $ %Hack for Math-highlighting
als Repository intialisieren.
\subsection{Einen \highlight{Branch} erstellen \& auf ihn wechseln}
Die Entwicklung auf zwei verschiedenen Branches verläuft komplett
unabhöngig voneinander
\begin{lstlisting}[language=bash]
  $ git checkout -b <NEW-BRANCHNAME>
\end{lstlisting} % $ %Hack for Math-highlighting
Wenn der Branch bereits existiert, kann die Option \lstinline{-b}
weggelassen werden.

Alle (lokalen) Branches kann man sich mit dem Befehl
\begin{lstlisting}[language=bash]
  $ git branch --list
\end{lstlisting} % $ %Hack for Math-highlighting
anzeigen lassen.
\subsection{Im Repository Arbeiten}
Wie auf einer nicht Versionierter Code-Base.

Ist das Arbeitsverzeichniss im Vergleich zum letzen Commit verändert,
so sagt man dass es \highlight{dirty}  also Schmutzig ist.
\subsection{Änderungen \highlight{Commiten}}
Die geänderten Dateien kann man sich mit dem Befehl 
\begin{lstlisting}[language=bash]
  $ git status
\end{lstlisting} % $ %Hack for Math-highlighting
ausgeben lassen.

Um seine Änderungen zum \highlight{Index} hinzuzufügen (also dafür zu
sorgen, dass Git sie sich merkt) nutzt man folgendes Kommando
\begin{lstlisting}[language=bash]
  $ git add <FILENAME>
\end{lstlisting} % $ %Hack for Math-highlighting
Dies funktioniert auch für Verzeichnisse. Gibt man als Dateinamen
\lstinline{.} an, so legt Git alle Änderungen auf den Index.

Man erstellt nun einen \highlight{Commit} mit dem Befehl 
\begin{lstlisting}[language=bash]
  $ git commit -m <COMMIT-NACHRICHT>
\end{lstlisting} % $ %Hack for Math-highlighting

\subsection{\highlight{Pushen} und \highlight{Pullen}}
Um Änderungen aus dem entfernten Repository zu laden:
\begin{lstlisting}[language=bash]
  $ git pull
\end{lstlisting} % $ %Hack for Math-highlighting

Um seine Commits auf das entfernte Repo zu laden:
\begin{lstlisting}[language=bash]
  $ git push
\end{lstlisting} % $ %Hack for Math-highlighting

\subsection{\highlight{Mergen}}
Git ist sehr gut darin, Änderungen zusammenzuführen, benötigt aber
manchmal dabei Hilfe (Git wird einen darauf hinweise). Diese
\highlight{Merge Konflikte} lassen sich mittels dem tool
\begin{lstlisting}[language=bash]
  $ git mergetool
\end{lstlisting} % $ %Hack for Math-highlighting
zusammenführen. Ich empfehle immer die Option
\lstinline{--tool=emerge} anzugeben\footnote{um seinen Merge mit dem
  besten Editor durchzuführen}.

\subsection{Branche zusammenführen}
Auch verschiedene Branches werden \enquote{gemerged}:
\begin{lstlisting}[language=bash]
  $ git checkout <ZIEL BRANCH>
  $ git merge <QUELL BRANCH>
\end{lstlisting} % $ %Hack for Math-highlighting
Merged den Branch \lstinline{<QUELL BRANCH>} in den Branch
\lstinline{<ZIEL BRANCH>}
\end{document}
